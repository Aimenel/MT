

\title{\bf{Scala and SBT}}
\author{
        	Nadja M\"uller
}
\date{\today}

\documentclass[12pt]{article}
\usepackage{listings}

% "define" Scala
\lstdefinelanguage{scala}{
  morekeywords={abstract,case,catch,class,def,%
    do,else,extends,false,final,finally,%
    for,if,implicit,import,match,mixin,%
    new,null,object,override,package,%
    private,protected,requires,return,sealed,%
    super,this,throw,trait,true,try,%
    type,val,var,while,with,yield},
  otherkeywords={=>,<-,<\%,<:,>:,\#,@},
  sensitive=true,
  morecomment=[l]{//},
  morecomment=[n]{/*}{*/},
  morestring=[b]",
  morestring=[b]',
  morestring=[b]"""
}

\begin{document}


\maketitle

\section{SBT Getting Started}
\subsection{directory}

\begin{lstlisting}
src/
  main/
    resources/
       <files to include in main jar here>
    scala/
       <main Scala sources>
    java/
       <main Java sources>
  test/
    resources
       <files to include in test jar here>
    scala/
       <test Scala sources>
    java/
       <test Java sources>
\end{lstlisting}


\section{Scala}

\subsection{Syntax}
\subsubsection{object}
Singleton object: a class with a single instance. This class is created on demand, the first time it is used.

\subsubsection{class}
Classes in Scala can have parameters.

\begin{lstlisting} [language=scala]


class Complex(real: Double, imaginary: Double) {
	...
	def re = real	
	def im = imaginary
	...
}

\end{lstlisting}

Cases classes:
\begin{itemize}
\item Var (String)
\item Const (Int)
\end{itemize}

A new keyword is not mandatory to create instances of a class.

\subsubsection{Method calls}

Method taking one argument can be used with an infix syntax.

\begin{lstlisting} [language=scala]
	def re() = real
	def im() = imaginary

	//method declaration without arguments
	def re = real
	def im = imaginary
\end{lstlisting}


\subsubsection{Interaction with Java}
All classes from the java.lang package are imported by default, while other need to be imported explicitly.

For example:
\begin{lstlisting} [language=scala]
//multiple classes
import java.util.{Date, Locale} 
//single class
import java.text.DateFormat 
//all classes in package
import java.text.DateFormat._ 

object FrenchDate {
	def main(args: Array[String]) {
		//creation of Java class Date
		val now = new Date 
		//method call
		val df = getDateInstance(LONG, Locale.FRANCE) 
		
		//infix format call		
		println(df format now)
		//equivalent method call		
		df.format(now)

	}
}
\end{lstlisting}

\subsection{Inheritance}
All classes in Scala inherit from a super-class. When no super-class is specified, $scala.AnyRef$ is implicitly used.

\subsubsection{Overriding}

\begin{lstlisting} [language=scala]
override def toString() = ...
\end{lstlisting}


\end{document}
