
\title{\bf{Appraoches}}

\documentclass[12pt]{article}

\begin{document}
\maketitle

\section{Predicate Quantifier}
New Definition:

\begin{equation}
	\mathbf{forall} \; x:T :: c(x) ==>\mathbf{acc}( pred(e_1(x), ... ,e_n(x)), p(x) )
\end{equation}

where {\it c(x)} is  a boolean expression,{\it e(x)} an injective reference-typed expression and {\it p(x)} a permission expression.

\subsection{Inverse Function}

\[ \mathbf{Given: \; } \mathbf{forall} \; x:T :: c(x) ==> \mathbf{acc}( pred(e_1(x), ... ,e_n(x)), p(x) )   \]
Let  $pred^{- 1}$ be a new funciton, where the following two inversion functions are true:

INV-1:
\[ \forall val_1, ..., val_n :: c(pred^{-1} (val_1, ..., val_n) => \]
\[ e_1 (pred^{-1} (val_1, ..., val_n)) == val_1 \: \&\&  ... \: \&\&  \:\: e_n(pred^{-1} (val_1, ..., val_n)) == val_n\]


INV-2:
\[ \forall x:T :: c(x) => pred^{-1}(e_1(x), ..., e_n(x)) == x\]

\subsection{Injectivity}
The mapping from input parameter to predicate has to be injective. Otherwise, there might not be a valid inverse function:


\begin{equation}
	\mathbf{forall} \; x:T :: c(x) ==> \mathbf{acc}( 1, ... ,1), p(x) ) ,
\end{equation}

There exists no inverse function for this example.

\subsection{Keeping track of known-folded permissions under predicate instances}

I would instantiate a predicate and its map separately at first use.

\end{document}
