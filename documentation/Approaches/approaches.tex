
\title{\bf{Appraoches}}

\documentclass[12pt]{article}
\usepackage{mathtools}

\begin{document}
\maketitle
\section{Pure Quantifiers}
Add trigger if needed and translate the Expression
\section{Fields}
\section{Predicate Quantifier}
New Definition:

\begin{equation}
	\mathbf{forall} \; x:T :: c(x) ==>\mathbf{acc}( pred(e_1(x), ... ,e_n(x)), p(x) )
\end{equation}

where {\it c(x)} is  a boolean expression,$pred(e_1(x), ... ,e_n(x))$ an injective reference expression and {\it p(x)} a permission expression.

\subsection{Inverse Function}

\[ \mathbf{Given: \; } \mathbf{forall} \; x:T :: c(x) \Rightarrow \mathbf{acc}( pred(e_1(x), ... ,e_n(x)), p(x) )   \]
Let  $pred^{- 1}$ be a new funciton, where the following two inversion functions are true:

INV-1:
\[ \forall val_1, ..., val_n :: c(pred^{-1} (val_1, ..., val_n) => \]
\[ e_1 (pred^{-1} (val_1, ..., val_n)) == val_1 \: \&\&  ... \: \&\&  \:\: e_n(pred^{-1} (val_1, ..., val_n)) == val_n\]


INV-2:
\[ \forall x:T :: c(x) => pred^{-1}(e_1(x), ..., e_n(x)) == x\]

\subsection{Injectivity}
The mapping from input parameter to predicate has to be injective. Otherwise, there might not be a valid inverse function:


\begin{equation}
	\mathbf{forall} \; x:T :: c(x) \Rightarrow \mathbf{acc}( 1, ... ,1), p(x) ) ,
\end{equation}

It sufficies that only one variable is not injective

There exists no inverse function for this example.

\section{Combinations}
A combination of \&\& will be desugared into two equations - or at least marginally treated as such.


A combination of or, implies will be transformed into an additional part of the condition c(x).

\subsection{and}
\subsubsection{pure and permission}
\begin{equation}
	\mathbf{forall} \; x:T :: c(x) \Rightarrow (b(x) \&\& \mathbf{acc}(e(x).f, p(x))
\end{equation}

Desugaring to:
\begin{equation}
\begin{aligned}
	\mathbf{forall} \; x:T :: c(x)  \Rightarrow b(x)\\
	\mathbf{forall} \; x:T :: c(x) \Rightarrow \mathbf{acc}(e(x).f, p(x))
\end{aligned}
\end{equation}

\subsubsection{permission and permission}

\begin{equation}
	\mathbf{forall} \; x:T :: c(x) \Rightarrow ( \mathbf{acc}(e_1(x).f, p_1(x) \&\& \mathbf{acc}(e_2(x).f, p_2(x) ))
\end{equation}

Desugaring to:
\begin{equation}
\begin{aligned}
	\mathbf{forall} \; x:T :: c(x) \Rightarrow \mathbf{acc}(e_1(x).f, p_1(x))\\
	\mathbf{forall} \; x:T :: c(x) \Rightarrow \mathbf{acc}(e_2(x).f, p_2(x))
\end{aligned}
\end{equation}

Implementing this by reusing parts from the before implementations should spare evaluating the same values twice and possibly defining a new Mask. Additionally the independent location description could be fused into one expression (in carbon).
%or
\subsection{or}
\begin{equation}
	\mathbf{forall} \; x:T :: c(x) \Rightarrow (b(x) \| \mathbf{acc}(e(x).f, p(x) ))
\end{equation}

Desugaring to:
\begin{equation}
	\mathbf{forall} \; x:T :: (c(x) \:\&\&\: not(b(x))) \Rightarrow  \mathbf{acc}(e(x).f, p(x))
\end{equation}

%implies
\subsection{implies}
\begin{equation}
	\mathbf{forall} \; x:T :: c(x) \Rightarrow (b(x) \Rightarrow \mathbf{acc}(e(x).f, p(x))
\end{equation}

Desugaring to:
\begin{equation}
	\mathbf{forall} \; x:T ::( c(x) \:\&\&\: b(x)) \Rightarrow \mathbf{acc}(e(x).f, p(x))
\end{equation}
\section{Nested Quantifiers}
With the rewriting mentioned above, there will be two ways which make sense to support nested quantifiers:
\begin{equation}
\begin{aligned}
	\mathbf{forall} \; x:T_1, y:T_2 :: c(x) \Rightarrow \mathbf{acc}(e_1(x, y).f, p(x, y))\\
	\mathbf{forall} \; x:T_1 :: c_1(x) \Rightarrow (\mathbf{forall} \; y:T_2 :: c_2(x, y) \Rightarrow \mathbf{acc}(e_2(x, y).f, p(x,y)))
\end{aligned}
\end{equation}

The second vairant can easily be rewritten to the first structure:
\begin{equation}
\begin{aligned}
	\mathbf{forall} \; x:T_1, y:T_2 :: (c_1(x)  \:\&\&\: c_2(x, y)) \Rightarrow \mathbf{acc}(e_1(x, y).f, p(x, y))\\
\end{aligned}
\end{equation}

For that case, the inverse functions would be:

\subsection{Injectivity Check}
each combination of parameter should point to different locations.
\begin{equation}
\begin{aligned}
	\mathbf{forall} \; v1_1:T_1, v2_1:T_1, ..., v1_n:T_n, v2_n:T_n  :: \\
	(c(v1_1,..., v1_n) \:\&\&\:  c(v2_1, v2_n)  \:\&\&\:  ((v1_1 != v1_1) \| ... \| (v_1n != v_2n)) \\
 	\Rightarrow \mathbf{acc} e_1(v1_1,..., v1_n)  != e_1(v2_1,..., v2_n) \\
\end{aligned}
\end{equation}

\subsection{Inverse Function}
Inverse Function 1:
\begin{equation}
\begin{aligned}
\mathbf{forall} \; v1:T_1,  v_n:T_n  :: (c(v_1,..., v_n) \\
\Rightarrow \mathbf{inv_1}(e(v_1,..., v_n))  == v_1 \\
... \\
\mathbf{forall} \; v1:T_1,  v_n:T_n  :: (c(v_1,..., v_n) \\
 \Rightarrow \mathbf{inv_n}(e(v_1,..., v_n))  == v_n \\
\end{aligned}
\end{equation}

Inverse Function 2:
\begin{equation}
\begin{aligned}
\mathbf{forall} \; o:Ref  :: (c( \mathbf{inv_1}(o), ...,  \mathbf{inv_n}(o)) \\
 \Rightarrow e(\mathbf{inv_1}(o))  == o \\
...\\
\mathbf{forall} \; o:Ref  :: (c( \mathbf{inv_1}(o), ...,  \mathbf{inv_n}(o))\\
 \Rightarrow e(\mathbf{inv_1}(o))  == o 
\end{aligned}
\end{equation}
\end{document}

